\documentclass[11pt,a4paper,oneside]{article}
%\theauthor - The author. To define the author, use \author{}.
%\theemail  - The author's email. To define the author's email, use \email{}.
%\thestatus - The status (usually draft or final). To define the status, use
%             \status{}.
%\thetitle  - The title. To define the title, use \title{1}
\usepackage{mhldockit}
\input{common}

\author{Michael Heilmann}
\email{\href{mailto:michaelheilmann@primordialmachine.com}{michaelheilmann@primordialmachine.com}}
\status{Draft}

\addbibresource{bibliography.bib}
\title{$\mathfrak{HLL}$ Language Specification}
\begin{document}
\maketitle
\begin{abstract}
$\mathfrak{HLL}$ is a set of lexical, syntactical and    semantical properties
of programming languages.            Any programming language which   fulfills
all those properties is called an $\mathfrak{HLL}$ language. The reason    for
grouping those properties is that other specifications can build       on this
specification.
\end{abstract}
\tableofcontents
%Some inspirations were drawn from various Sather dialects most notably
%GNU Sather (\cite{gnu-sather}).

\section{Operators}

\begin{table}[htbp]
\small
\centering
\begin{tabular}{ll}
\textbf{Group}           &\textbf{Operator}                   \\
[1ex]\hline\\[-1.5ex]
Parentheses              &\texttt{(}\textit{a}\texttt{)}      \\
[1ex]\hline\\[-1.5ex]
Access Operator          &\textit{a}\texttt{.}\textit{b}      \\
[1ex]\hline\\[-1.5ex]
Exponentiation           &\textit{a} \texttt{**} \textit{b}   \\
[1ex]\hline\\[-1.5ex]
Multiplication           &\textit{a} \texttt{*} \textit{b}    \\
                         &\textit{a} \texttt{/} \textit{b}    \\
                         &\textit{a} \texttt{mod} \textit{b}  \\
                         &\textit{a} \texttt{rem} \textit{b}  \\
[1ex]\hline\\[-1.5ex]
Additive Identity and    &\texttt{+} \textit{a}               \\
Negation                 &\texttt{-} \textit{a}               \\
[1ex]\hline\\[-1.5ex]
Addition                 &\textit{a} \texttt{+} \textit{b}    \\
                         &\textit{a} \texttt{-} \textit{b}    \\
[1ex]\hline\\[-1.5ex]
Relational Operators     &\textit{a} \texttt{<} \textit{b}    \\
                         &\textit{a} \texttt{<=} \textit{b}   \\
                         &\textit{a} \texttt{>} \textit{b}    \\
                         &\textit{a} \texttt{>=} \textit{b}   \\
[1ex]\hline\\[-1.5ex]
Equality Operators       &\textit{a} \texttt{/=} \textit{b}   \\
                         &\textit{a} \texttt{=} \textit{b}    \\
[1ex]\hline\\[-1.5ex]
Logical Conjunction      &\texttt{and} \textit{a}             \\
[1ex]\hline\\[-1.5ex]
Logical Disjunction      &\texttt{or} \textit{a}              \\
[1ex]\hline\\[-1.5ex]
Logical Negation         &\texttt{not} \textit{a}             \\
[1ex]\hline\\[-1.5ex]
%[1ex]\hline\\[-1.5ex]
%Assignment Operator      &\textit{a} \texttt{:=} \textit{b}   \\
\end{tabular}
\label{611a28d0-3a66-4073-a89c-0b9a0b50a646}
\caption{Expression Operator Precedence Groups.}
\end{table}

\subsubsection{Statement operators}

\input{lexical_grammar}
\input{syntactical_grammar}

\begin{figure}
\centering
\begin{tikzpicture}
  \node (is-root) {$+$}
                    child { node {$e_1$} }
                    child [missing]
                    child { node {$e_2$} }
  ;
\end{tikzpicture}
\caption{AST-prototype for \texttt{add}.}
\end{figure}

\begin{figure}
\centering
\begin{tikzpicture}
  \node (is-root) {$-$}
                    child { node {$e_1$} }
                    child [missing]
                    child { node {$e_2$} }
  ;
\end{tikzpicture}
\caption{AST-prototype for \texttt{sub}.}
\end{figure}

%\bibliographystyle{plainnat}
\printbibliography

\end{document}
